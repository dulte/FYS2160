\documentclass[a4paper,norsk, 10pt]{article}
\usepackage[utf8]{inputenc}
\usepackage{verbatim}
\usepackage{listings}
\usepackage{graphicx}
\usepackage[norsk]{babel}
\usepackage{a4wide}
\usepackage{color}
\usepackage{amsmath}
\usepackage{float}
\usepackage{amssymb}
\usepackage[dvips]{epsfig}
\usepackage[toc,page]{appendix}
\usepackage[T1]{fontenc}
\usepackage{cite} % [2,3,4] --> [2--4]
\usepackage{shadow}
\usepackage{hyperref}
\usepackage{titling}
\usepackage{marvosym }
\usepackage{subcaption}
\usepackage[noabbrev]{cleveref}
\usepackage{cite}


\setlength{\droptitle}{-10em}   % This is your set screw

\setcounter{tocdepth}{2}

\newcommand{\pd}[2]{\frac{\partial #1}{\partial #2}}

\lstset{language=c++}
\lstset{alsolanguage=[90]Fortran}
\lstset{alsolanguage=Python}
\lstset{basicstyle=\small}
\lstset{backgroundcolor=\color{white}}
\lstset{frame=single}
\lstset{stringstyle=\ttfamily}
\lstset{keywordstyle=\color{red}\bfseries}
\lstset{commentstyle=\itshape\color{blue}}
\lstset{showspaces=false}
\lstset{showstringspaces=false}
\lstset{showtabs=false}
\lstset{breaklines}
\title{Fys2160 Oblig 2}
\author{Daniel Heinesen, daniehei}
\begin{document}
\maketitle

\section{Exercise 1)}

\subsection{a)}
The partition function is given as

\begin{equation}
Z = \sum_s e^{-E(s)/kT} = \sum_s e^{-E(s)\beta}
\end{equation}\label{eq:partition}

We have the energies $\epsilon_1 = \epsilon$ and $\epsilon_2 = \epsilon_3 = \epsilon_4 = 2\epsilon$, giving us the partition function:

\begin{equation}
Z = e^{-\beta \epsilon} + 3e^{-2\beta \epsilon}
\end{equation}

\subsection{b)}
From the partition function we are able to find the average energy from the following equation:

\begin{equation}
\langle E \rangle = - \pd{\ln Z}{\beta}
\end{equation}

We can from this find the average energy for our system:

\begin{equation}
\langle E \rangle = -\pd{}{\beta} \ln \left(e^{-\beta\epsilon} +3e^{-2\epsilon\beta}\right)
\end{equation}

\begin{equation}
= -\frac{-\epsilon e^{-\beta\epsilon} - 6\epsilon e^{-2\beta\epsilon}}{e^{-\beta\epsilon} +3e^{-2\epsilon\beta}}
= \epsilon\frac{e^{\beta\epsilon} + 6}{e^{\beta\epsilon} +3}
\end{equation}

\end{document}



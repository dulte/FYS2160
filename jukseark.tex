\documentclass[a4paper,norsk, 10pt]{article}
\usepackage[utf8]{inputenc}
\usepackage{verbatim}
\usepackage{listings}
\usepackage{graphicx}
\usepackage[norsk]{babel}
\usepackage{a4wide}
\usepackage{color}
\usepackage{amsmath}
\usepackage{float}
\usepackage{amssymb}
\usepackage[dvips]{epsfig}
\usepackage[toc,page]{appendix}
\usepackage[T1]{fontenc}
\usepackage{cite} % [2,3,4] --> [2--4]
\usepackage{shadow}
\usepackage{hyperref}
\usepackage{titling}
\usepackage{marvosym }
\usepackage{subcaption}
\usepackage[noabbrev]{cleveref}
\usepackage{cite}


\setlength{\droptitle}{-10em}   % This is your set screw

\setcounter{tocdepth}{2}

\lstset{language=c++}
\lstset{alsolanguage=[90]Fortran}
\lstset{alsolanguage=Python}
\lstset{basicstyle=\small}
\lstset{backgroundcolor=\color{white}}
\lstset{frame=single}
\lstset{stringstyle=\ttfamily}
\lstset{keywordstyle=\color{red}\bfseries}
\lstset{commentstyle=\itshape\color{blue}}
\lstset{showspaces=false}
\lstset{showstringspaces=false}
\lstset{showtabs=false}
\lstset{breaklines}
\title{FYS3140 Oblig1}
\author{Daniel Heinesen, daniehei}
\begin{document}
%\maketitle
\section{Energy in Thermal Dynamics}
\begin{table}[H]
\centering
\begin{tabular}{c c}
Exchanged Quantity & Type of Equilibrium\\
\hline
energy & thermal\\
volume & mechanical \\
particles & diffusive
\end{tabular}
\end{table}

\subsection{Ideal Gas}
Ideal gas law
\begin{equation}
PV = nRT = NkT
\end{equation}

\begin{equation}
N = nN_A, \qquad R = 8.31 J/mol\cdot K, \qquad N_A = 6.02\cdot10^{23} \qquad k =  R/N_A = 1.381\cdot 10^{-23} J/K
\end{equation}

\subsubsection{Microscopical Model}
\begin{equation}
\bar{P} = \frac{\bar{F}_{\text{x, on piston}}}{A} = \frac{-\bar{F}_{\text{x, on particle}}}{A} = -\frac{m\bar{\left(\frac{\Delta v_x}{\Delta t}\right)}}{A} = \frac{mv_x^2}{V}
\end{equation}
\begin{equation}
\Delta t = 2L/v_x, \qquad \Delta v_x = -2v_x
\end{equation}
\begin{equation}
PV = Nm\bar{v}_x^2 \Rightarrow \overline{\frac{1}{2}mv_x^2} = \frac{1}{2}kT \Rightarrow \overline{K}_{trans} = 3\cdot\frac{1}{2}kT
\end{equation}
\begin{equation}
v_{rms} = \sqrt{\bar{v^2}} = \sqrt{\frac{3kT}{m}}
\end{equation}
\subsection{Equipartition Theorem}
\begin{equation}
U_{thermal} = N\cdot f\cdot\frac{1}{2}kT
\end{equation}
\subsection{Heat and Work}
\textbf{Temperature}: measure of the tendency of an object to spontaneously give up energy to its surroundings.
\textbf{Heat}: any spontaneous flow of energy from one object to another, caused by a difference in temperatures.
\textbf{Work}: any other transfer of energy in or out of the system.
\begin{equation}
\Delta U = Q + W
\end{equation}
\subsection{Compression Work}
\begin{equation}
W = -P\Delta V \text{ (for quasistatic compression)}
\end{equation}
With $P(V)$
\begin{equation}
W = -\int_{V_i}^{V_f}P(V)dV
\end{equation}
\subsubsection{Compression of Ideal Gas}
\begin{equation}
W = NkT\ln\frac{V_i}{V_f} \Rightarrow Q = \Delta U - W = \Delta (1/2 NfkT) - W = W = NkT\ln\frac{V_f}{V_i}
\end{equation}
For adiabatic compression:
\begin{equation}
\Delta U = W \Rightarrow dU = \frac{1}{2}fnkdT = -PdV \Rightarrow \frac{f}{2}\frac{dT}{T} = -\frac{dV}{V}
\end{equation}
\begin{equation}
V_f T_f^{f/2} = V_i T_i^{f/2} = \text{constant}, \qquad VT^{f/2} = \text{constant},\qquad V^\gamma P = \text{constant}
\end{equation}
$\gamma = (f+2)/f$ is the adiabatic exponent.

\subsection{Heat Capacity}
\begin{equation}
C = \frac{ Q}{\Delta T} \text{ (heat capacity)}, \qquad c = \frac{C}{m} \text{ (specific heat capacity})
\end{equation}
\begin{equation}
C = \frac{ Q}{\Delta T} = C = \frac{\Delta U - W}{\Delta T}
\end{equation}
\begin{equation}
C_V =\left( \frac{\partial U}{\partial T}\right)_V, \qquad C_P = \left(\frac{\Delta U -(-P\Delta V)}{\Delta T}\right)_P = \left( \frac{\partial U}{\partial T}\right)_P + P\left( \frac{\partial V}{\partial T}\right)_P
\end{equation}
\subsubsection{For Ideal Gas}
\begin{equation}
C_V = \frac{\partial}{\partial T }\frac{NfkT}{2} = \frac{Nfk}{2}, \qquad \left(\frac{\partial V}{\partial T}\right)_P = \frac{\partial}{\partial T }\frac{NkT}{P} = \frac{Nk}{P} \Rightarrow C_P = C_V + Nk = C_V + nR
\end{equation}
rule of Dulong And Petit: heat capacity of solid should go towards $3R$

\subsection{Latent Heat}
For phase transformation
\begin{equation}
L = \frac{Q}{m}
\end{equation}
To accomplish the transformation.
\subsection{Enthalpy}
Total energy one has to come up with to create the system and put it into the environment 
\begin{equation}
H = U + PV
\end{equation}
\begin{equation}
\Delta H = \Delta U + P\Delta V = Q +W_{other} \text{ (constant P)}
\end{equation}
\begin{equation}
C_P = \left(\frac{\partial H}{\partial T}\right)_P
\end{equation}

\subsection{Rates of Processes}
\subsubsection{Heat Conduction}
\begin{equation}
Q \propto \frac{A\Delta T\Delta t}{\Delta x} \Rightarrow \frac{Q}{\Delta t} = -k_t A \frac{dT}{dx}
\end{equation}
Fourier heat conduction law
\subsubsection{Conductivity of Idea Gas}
\begin{equation}
\ell \approx \frac{1}{4\pi r^2}\frac{V}{N},\qquad Q = -\frac{1}{2}C_V\ell \frac{dT}{dx}, \qquad k_t = \frac{1}{2}\frac{C_V}{V}\ell \bar{v}
\end{equation}
\begin{equation}
\bar{v}\propto \sqrt{T}
\end{equation}

\subsubsection{Viscosity}
\begin{equation}
F_x \propto \frac{A\cdot (u_{x,top}- u_{x,bottom})}{\Delta z} \Rightarrow \frac{|F_x|}{A} = \eta \frac{du_x}{dz}
\end{equation}

\subsubsection{Diffusion}
\begin{equation}
J_x = -D\frac{dn}{dx}
\end{equation}
$J_x$, flux has units number of particles per unit area per unit time.

\section{The Second Law}
\subsection{Two-State System}
\begin{equation}
\text{probability of n heads} = \frac{\Omega(n)}{\Omega(all)}
\end{equation}
\begin{equation}
\Omega(N,n) = \frac{N!}{n!(N-n)!} = \binom{N}{n}
\end{equation}
For paramagnet
\begin{equation}
\Omega(N_{\uparrow}) = \binom{N}{N_{\uparrow}} = \frac{N!}{N_\uparrow ! N_\downarrow !}
\end{equation}
For Einstein Solid
\begin{equation}
\Omega(N,q) = \binom{q+N-1}{q}=\frac{(q+N-1)!}{q!(N-1)!}
\end{equation}
$N$ is oscillators, $q$ is energy units.

\subsection{Interacting Systems}
\begin{equation}
N_A = N_B, \qquad q_{total} = q_A + q_B
\end{equation}
\textbf{Fundamental assumption of statistical mechanics}: In an isolated system in thermal equilibrium, all accessible microstates are equally probable.

\subsection{Stirling's Approximation}
\begin{equation}
N!\approx N^Ne^{-N}\sqrt{2\pi N},\qquad \ln N! \approx N\ln N - N
\end{equation}

\subsection{Multiplicity of a Large Einstein Solid}
\begin{equation}
\Omega(N,q) \approx \frac{(q+N)!}{q!N!}
\end{equation}
\begin{equation}
\ln\Omega \approx N\ln\frac{q}{N} + N + \frac{N^2}{q}
\end{equation}
(Remember to use $\ln(x+1) \approx x$.)
\begin{equation}
\Rightarrow \Omega(N,q) \approx e^{N\ln(q/N)}e^N = \left(\frac{eq}{N}\right)^N,\qquad q >>N
\end{equation}
\begin{equation}
\text{width of peak} = \frac{q}{\sqrt{N}}
\end{equation}
\subsection{Ideal Gas}
\begin{equation}
\Omega_1 \propto V\cdot V_p,\qquad 2mU = p^2_x + p^2_y + p^2_z,\qquad \Delta x \Delta p \approx h 
\end{equation}
\begin{equation}
\Omega_N = \frac{1}{N!}\frac{V^N}{h^{3N}}\cdot A_{hypersphere},\qquad A_{hypersphere} = \frac{2\pi^{d/2}}{(d/2 - 1)!}r^{r-1}
\end{equation}
\begin{equation}
\Omega(U,V,N) = f(N)V^NU^{3N/2}
\end{equation}
\subsubsection{Interacting Ideal Gas}
\begin{equation}
\Omega_{total} = (f(N))^2(V_AV_B)^2(U_AU_B)^{3N/2}
\end{equation}
\begin{equation}
\text{width of peak} = \frac{U_{total}}{\sqrt{3N/2}}
\end{equation}
If can exchange volume:
\begin{equation}
\text{width of peak} = \frac{V_{total}}{\sqrt{N}}
\end{equation}

\subsection{Entropy}
\begin{equation}
S = k\ln\Omega
\end{equation}
\subsection{Entropy of Ideal Gas}
Monatomic ideal gas, Sackur-Tetrode eq:
\begin{equation}
S = Nk\left[\ln\left(\frac{V}{N}\left(\frac{4\pi mU}{3Nh^2}\right)^{3/2}\right) + \frac{5}{2}\right]
\end{equation}
For $U$, $N$ fixed:
\begin{equation}
\Delta S = Nk\ln\frac{V_f}{V_i}
\end{equation}
\section{Interactions and Implications}
\begin{equation}
\frac{\partial S_A}{\partial U_A} = \frac{\partial S_B}{\partial U_B}
\end{equation}
at equilibrium.
\begin{equation}
\frac{1}{T}\equiv \left(\frac{\partial S}{\partial U}\right)_{N,V}
\end{equation}
\subsection{Entropy and Heat}
\subsubsection{Predicting Heat capacity}
\begin{equation}
C_V = \left(\frac{\partial U}{\partial T}\right)_{N,V}
\end{equation}
Algorithm:
\begin{itemize}
\item Use QM and some combinations to find an expression for $\Omega$, in terms of $U$, $V$ and $N$, and any other relevant variables
\item Take to logarithm to find $S$
\item Differentiate $S$ with respect to $U$ and the the reciprocal to find the temperature $T$ as a function of $U$ and other variables.
\end{itemize}

\subsubsection{Measuring Entropies}
For constant(or quatistatic) volume and no work
\begin{equation}
dS = \frac{dU}{T} = \frac{Q}{T}
\end{equation}
More general
\begin{equation}
dS = \frac{C_V dT}{T},\qquad \Delta S = \int_{T_i}^{T_f}\frac{C_V}{T}dT,\qquad S - S(0) = \int_{T_i}^{0}\frac{C_V}{T}dT
\end{equation}
\textbf{Third law}: $T\rightarrow 0 \Rightarrow S\rightarrow0$
\subsection{Paramagnetism}
\begin{equation}
U = \mu B(N_\downarrow - N_\uparrow) = \mu B (N- 2N_\uparrow),\qquad M = \mu(N_\uparrow - N_\downarrow) = -\frac{U}{B}
\end{equation}
\subsubsection{Analytic Solution}
\begin{equation}
S/k \approx N\ln N - N_\uparrow \ln N_\uparrow - (N-N_\uparrow)\ln(N-N_\uparrow)
\end{equation}
\begin{equation}
\frac{1}{T} = \frac{k}{2\mu B}\ln\left(\frac{N- U/\mu B}{N+ U/\mu B}\right)
\end{equation}
\begin{equation}
U = N\mu B\left(\frac{1-e^{2\mu B/kT}}{1+e^{2\mu B/kT}}\right) = -N\mu B\tanh\frac{\mu B}{kT},\qquad M = N\mu \tanh\frac{\mu B}{kT}
\end{equation}
\begin{equation}
C_B = \left(\frac{\partial U}{\partial T}\right)_{N,B} = Nk \frac{(\mu B/kT)^2}{\cosh^2(\mu B/kT	)}
\end{equation}
Bohr magnetron
\begin{equation}
\mu_B = \frac{eh}{4\pi m_e} = 9.274\cdot10^{-24} J/T = 5.788\cdot10^{-5}eV/T
\end{equation}
For $\mu B/kT << 1$
\begin{equation}
M\approx \frac{N\mu^2B}{kT} \Rightarrow M \propto 1/T
\end{equation}
Curie's law.
\subsection{Summery}
Thermodynamic identity
\begin{equation}
dU = TdS - PdV + \mu dN
\end{equation}
$VNP:$
\begin{table}[H]
\centering
\begin{tabular}{c c c c c}
Type of interaction & Exchange quantity & Governing variable & Constant & Formula\\
\hline
thermal & energy & temperature & V,N & $\frac{1}{T} = \left(\frac{\partial S}{\partial U}\right)_{V,N}$\\
mechanical & volume & pressure & U,N & $\frac{P}{T} = \left(\frac{\partial S}{\partial V}\right)_{U,N}$\\
diffusive & particles & chemical potential & U,V & $\frac{\mu}{T} = -\left(\frac{\partial S}{\partial N}\right)_{U,V}$
\end{tabular}
\end{table}
\section{Engines and Refrigerators}
\subsection{Heat Engines}
efficiency
\begin{equation}
e \equiv \frac{\text{benefit}}{\text{cost}} = \frac{W}{Q_h}
\end{equation}
$Q_h$ is heat from the hot reservoir with temperature $T_h$, and $Q_c$ from the cold reservoir with temperature $T_c$.
\begin{equation}
Q_h = Q_c + W, \qquad e = 1 - \frac{Q_c}{Q_h}
\end{equation}
From second law
\begin{equation}
S_c \geq S_h \Rightarrow \frac{Q_c}{T_c} \geq \frac{Q_h}{T_h} \Rightarrow \frac{Q_c}{Q_h} \geq \frac{T_c}{T_h}
\label{eq:heatSecond}
\end{equation}
\begin{equation}
 \Rightarrow e \leq 1 - \frac{T_c}{T_h}
\end{equation}
\subsection{Refrigerators}
coefficient of preference:
\begin{equation}
COP \equiv \frac{\text{benefit}}{\text{cost}} = \frac{Q_c}{W}
\end{equation}
From first law $Q_h = Q_c + W$ we get
\begin{equation}
COP = \frac{Q_c}{Q_h - Q_c} = \frac{1}{Q_h/Q_c - 1}
\end{equation}
From second law \eqref{eq:heatSecond} we get
\begin{equation}
COP \leq \frac{1}{T_h/T_c - 1} = \frac{T_c}{T_h - T_c}
\end{equation}

\end{document}


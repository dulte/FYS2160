\documentclass[a4paper,norsk, 10pt]{article}
\usepackage[utf8]{inputenc}
\usepackage{verbatim}
\usepackage{listings}
\usepackage{graphicx}
\usepackage[norsk]{babel}
\usepackage{a4wide}
\usepackage{color}
\usepackage{amsmath}
\usepackage{float}
\usepackage{amssymb}
\usepackage[dvips]{epsfig}
\usepackage[toc,page]{appendix}
\usepackage[T1]{fontenc}
\usepackage{cite} % [2,3,4] --> [2--4]
\usepackage{shadow}
\usepackage{hyperref}
\usepackage{titling}
\usepackage{marvosym }
\usepackage{subcaption}
\usepackage[noabbrev]{cleveref}
\usepackage{cite}

\newcommand{\pd}[3]{\left(\frac{\partial #1}{\partial #2}\right)_{#3}}

\setlength{\droptitle}{-10em}   % This is your set screw

\setcounter{tocdepth}{2}

\lstset{language=c++}
\lstset{alsolanguage=[90]Fortran}
\lstset{alsolanguage=Python}
\lstset{basicstyle=\small}
\lstset{backgroundcolor=\color{white}}
\lstset{frame=single}
\lstset{stringstyle=\ttfamily}
\lstset{keywordstyle=\color{red}\bfseries}
\lstset{commentstyle=\itshape\color{blue}}
\lstset{showspaces=false}
\lstset{showstringspaces=false}
\lstset{showtabs=false}
\lstset{breaklines}
\title{FYS3140 Oblig1}
\author{Daniel Heinesen, daniehei}
\begin{document}
%\maketitle
\section{Energy in Thermal Dynamics}
\begin{table}[H]
\centering
\begin{tabular}{c c}
Exchanged Quantity & Type of Equilibrium\\
\hline
energy & thermal\\
volume & mechanical \\
particles & diffusive
\end{tabular}
\end{table}

\subsection{Ideal Gas}
Ideal gas law
\begin{equation}
PV = nRT = NkT
\end{equation}

\begin{equation}
N = nN_A, \qquad R = 8.31 J/mol\cdot K, \qquad N_A = 6.02\cdot10^{23} \qquad k =  R/N_A = 1.381\cdot 10^{-23} J/K
\end{equation}

\subsubsection{Microscopical Model}
\begin{equation}
\bar{P} = \frac{\bar{F}_{\text{x, on piston}}}{A} = \frac{-\bar{F}_{\text{x, on particle}}}{A} = -\frac{m\bar{\left(\frac{\Delta v_x}{\Delta t}\right)}}{A} = \frac{mv_x^2}{V}
\end{equation}
\begin{equation}
\Delta t = 2L/v_x, \qquad \Delta v_x = -2v_x
\end{equation}
\begin{equation}
PV = Nm\bar{v}_x^2 \Rightarrow \overline{\frac{1}{2}mv_x^2} = \frac{1}{2}kT \Rightarrow \overline{K}_{trans} = 3\cdot\frac{1}{2}kT
\end{equation}
\begin{equation}
v_{rms} = \sqrt{\bar{v^2}} = \sqrt{\frac{3kT}{m}}
\end{equation}
\subsection{Equipartition Theorem}
\begin{equation}
U_{thermal} = N\cdot f\cdot\frac{1}{2}kT
\end{equation}
\subsection{Heat and Work}
\textbf{Temperature}: measure of the tendency of an object to spontaneously give up energy to its surroundings.
\textbf{Heat}: any spontaneous flow of energy from one object to another, caused by a difference in temperatures.
\textbf{Work}: any other transfer of energy in or out of the system.
\begin{equation}
\Delta U = Q + W
\end{equation}
\subsection{Compression Work}
\begin{equation}
W = -P\Delta V \text{ (for quasistatic compression)}
\end{equation}
With $P(V)$
\begin{equation}
W = -\int_{V_i}^{V_f}P(V)dV
\end{equation}
\subsubsection{Compression of Ideal Gas}
\begin{equation}
W = NkT\ln\frac{V_i}{V_f} \Rightarrow Q = \Delta U - W = \Delta (1/2 NfkT) - W = W = NkT\ln\frac{V_f}{V_i}
\end{equation}
For adiabatic compression:
\begin{equation}
\Delta U = W \Rightarrow dU = \frac{1}{2}fnkdT = -PdV \Rightarrow \frac{f}{2}\frac{dT}{T} = -\frac{dV}{V}
\end{equation}
\begin{equation}
V_f T_f^{f/2} = V_i T_i^{f/2} = \text{constant}, \qquad VT^{f/2} = \text{constant},\qquad V^\gamma P = \text{constant}
\end{equation}
$\gamma = (f+2)/f$ is the adiabatic exponent.

\subsection{Heat Capacity}
\begin{equation}
C = \frac{ Q}{\Delta T} \text{ (heat capacity)}, \qquad c = \frac{C}{m} \text{ (specific heat capacity})
\end{equation}
\begin{equation}
C = \frac{ Q}{\Delta T} = C = \frac{\Delta U - W}{\Delta T}
\end{equation}
\begin{equation}
C_V =\left( \frac{\partial U}{\partial T}\right)_V, \qquad C_P = \left(\frac{\Delta U -(-P\Delta V)}{\Delta T}\right)_P = \left( \frac{\partial U}{\partial T}\right)_P + P\left( \frac{\partial V}{\partial T}\right)_P
\end{equation}
\subsubsection{For Ideal Gas}
\begin{equation}
C_V = \frac{\partial}{\partial T }\frac{NfkT}{2} = \frac{Nfk}{2}, \qquad \left(\frac{\partial V}{\partial T}\right)_P = \frac{\partial}{\partial T }\frac{NkT}{P} = \frac{Nk}{P} \Rightarrow C_P = C_V + Nk = C_V + nR
\end{equation}
rule of Dulong And Petit: heat capacity of solid should go towards $3R$

\subsection{Latent Heat}
For phase transformation
\begin{equation}
L = \frac{Q}{m}
\end{equation}
To accomplish the transformation.
\subsection{Enthalpy}
Total energy one has to come up with to create the system and put it into the environment 
\begin{equation}
H = U + PV
\end{equation}
\begin{equation}
\Delta H = \Delta U + P\Delta V = Q +W_{other} \text{ (constant P)}
\end{equation}
\begin{equation}
C_P = \left(\frac{\partial H}{\partial T}\right)_P
\end{equation}

\subsection{Rates of Processes}
\subsubsection{Heat Conduction}
\begin{equation}
Q \propto \frac{A\Delta T\Delta t}{\Delta x} \Rightarrow \frac{Q}{\Delta t} = -k_t A \frac{dT}{dx}
\end{equation}
Fourier heat conduction law
\subsubsection{Conductivity of Idea Gas}
\begin{equation}
\ell \approx \frac{1}{4\pi r^2}\frac{V}{N},\qquad Q = -\frac{1}{2}C_V\ell \frac{dT}{dx}, \qquad k_t = \frac{1}{2}\frac{C_V}{V}\ell \bar{v}
\end{equation}
\begin{equation}
\bar{v}\propto \sqrt{T}
\end{equation}

\subsubsection{Viscosity}
\begin{equation}
F_x \propto \frac{A\cdot (u_{x,top}- u_{x,bottom})}{\Delta z} \Rightarrow \frac{|F_x|}{A} = \eta \frac{du_x}{dz}
\end{equation}

\subsubsection{Diffusion}
\begin{equation}
J_x = -D\frac{dn}{dx}
\end{equation}
$J_x$, flux has units number of particles per unit area per unit time.

\section{The Second Law}
\subsection{Two-State System}
\begin{equation}
\text{probability of n heads} = \frac{\Omega(n)}{\Omega(all)}
\end{equation}
\begin{equation}
\Omega(N,n) = \frac{N!}{n!(N-n)!} = \binom{N}{n}
\end{equation}
For paramagnet
\begin{equation}
\Omega(N_{\uparrow}) = \binom{N}{N_{\uparrow}} = \frac{N!}{N_\uparrow ! N_\downarrow !}
\end{equation}
For Einstein Solid
\begin{equation}
\Omega(N,q) = \binom{q+N-1}{q}=\frac{(q+N-1)!}{q!(N-1)!}
\end{equation}
$N$ is oscillators, $q$ is energy units.

\subsection{Interacting Systems}
\begin{equation}
N_A = N_B, \qquad q_{total} = q_A + q_B
\end{equation}
\textbf{Fundamental assumption of statistical mechanics}: In an isolated system in thermal equilibrium, all accessible microstates are equally probable.

\subsection{Stirling's Approximation}
\begin{equation}
N!\approx N^Ne^{-N}\sqrt{2\pi N},\qquad \ln N! \approx N\ln N - N
\end{equation}

\subsection{Multiplicity of a Large Einstein Solid}
\begin{equation}
\Omega(N,q) \approx \frac{(q+N)!}{q!N!}
\end{equation}
\begin{equation}
\ln\Omega \approx N\ln\frac{q}{N} + N + \frac{N^2}{q}
\end{equation}
(Remember to use $\ln(x+1) \approx x$.)
\begin{equation}
\Rightarrow \Omega(N,q) \approx e^{N\ln(q/N)}e^N = \left(\frac{eq}{N}\right)^N,\qquad q >>N
\end{equation}
\begin{equation}
\text{width of peak} = \frac{q}{\sqrt{N}}
\end{equation}
\subsection{Ideal Gas}
\begin{equation}
\Omega_1 \propto V\cdot V_p,\qquad 2mU = p^2_x + p^2_y + p^2_z,\qquad \Delta x \Delta p \approx h 
\end{equation}
\begin{equation}
\Omega_N = \frac{1}{N!}\frac{V^N}{h^{3N}}\cdot A_{hypersphere},\qquad A_{hypersphere} = \frac{2\pi^{d/2}}{(d/2 - 1)!}r^{r-1}
\end{equation}
\begin{equation}
\Omega(U,V,N) = f(N)V^NU^{3N/2}
\end{equation}
\subsubsection{Interacting Ideal Gas}
\begin{equation}
\Omega_{total} = (f(N))^2(V_AV_B)^2(U_AU_B)^{3N/2}
\end{equation}
\begin{equation}
\text{width of peak} = \frac{U_{total}}{\sqrt{3N/2}}
\end{equation}
If can exchange volume:
\begin{equation}
\text{width of peak} = \frac{V_{total}}{\sqrt{N}}
\end{equation}

\subsection{Entropy}
\begin{equation}
S = k\ln\Omega
\end{equation}
\subsection{Entropy of Ideal Gas}
Monatomic ideal gas, Sackur-Tetrode eq:
\begin{equation}
S = Nk\left[\ln\left(\frac{V}{N}\left(\frac{4\pi mU}{3Nh^2}\right)^{3/2}\right) + \frac{5}{2}\right]
\end{equation}
For $U$, $N$ fixed:
\begin{equation}
\Delta S = Nk\ln\frac{V_f}{V_i}
\end{equation}
\section{Interactions and Implications}
\begin{equation}
\frac{\partial S_A}{\partial U_A} = \frac{\partial S_B}{\partial U_B}
\end{equation}
at equilibrium.
\begin{equation}
\frac{1}{T}\equiv \left(\frac{\partial S}{\partial U}\right)_{N,V}
\end{equation}
\subsection{Entropy and Heat}
\subsubsection{Predicting Heat capacity}
\begin{equation}
C_V = \left(\frac{\partial U}{\partial T}\right)_{N,V}
\end{equation}
Algorithm:
\begin{itemize}
\item Use QM and some combinations to find an expression for $\Omega$, in terms of $U$, $V$ and $N$, and any other relevant variables
\item Take to logarithm to find $S$
\item Differentiate $S$ with respect to $U$ and the the reciprocal to find the temperature $T$ as a function of $U$ and other variables.
\end{itemize}

\subsubsection{Measuring Entropies}
For constant(or quatistatic) volume and no work
\begin{equation}
dS = \frac{dU}{T} = \frac{Q}{T}
\end{equation}
More general
\begin{equation}
dS = \frac{C_V dT}{T},\qquad \Delta S = \int_{T_i}^{T_f}\frac{C_V}{T}dT,\qquad S - S(0) = \int_{T_i}^{0}\frac{C_V}{T}dT
\end{equation}
\textbf{Third law}: $T\rightarrow 0 \Rightarrow S\rightarrow0$
\subsection{Paramagnetism}
\begin{equation}
U = \mu B(N_\downarrow - N_\uparrow) = \mu B (N- 2N_\uparrow),\qquad M = \mu(N_\uparrow - N_\downarrow) = -\frac{U}{B}
\end{equation}
\subsubsection{Analytic Solution}
\begin{equation}
S/k \approx N\ln N - N_\uparrow \ln N_\uparrow - (N-N_\uparrow)\ln(N-N_\uparrow)
\end{equation}
\begin{equation}
\frac{1}{T} = \frac{k}{2\mu B}\ln\left(\frac{N- U/\mu B}{N+ U/\mu B}\right)
\end{equation}
\begin{equation}
U = N\mu B\left(\frac{1-e^{2\mu B/kT}}{1+e^{2\mu B/kT}}\right) = -N\mu B\tanh\frac{\mu B}{kT},\qquad M = N\mu \tanh\frac{\mu B}{kT}
\end{equation}
\begin{equation}
C_B = \left(\frac{\partial U}{\partial T}\right)_{N,B} = Nk \frac{(\mu B/kT)^2}{\cosh^2(\mu B/kT	)}
\end{equation}
Bohr magnetron
\begin{equation}
\mu_B = \frac{eh}{4\pi m_e} = 9.274\cdot10^{-24} J/T = 5.788\cdot10^{-5}eV/T
\end{equation}
For $\mu B/kT << 1$
\begin{equation}
M\approx \frac{N\mu^2B}{kT} \Rightarrow M \propto 1/T
\end{equation}
Curie's law.
\subsection{Summery}
Thermodynamic identity
\begin{equation}
dU = TdS - PdV + \mu dN
\end{equation}
$VNP:$
\begin{table}[H]
\centering
\begin{tabular}{c c c c c}
Type of interaction & Exchange quantity & Governing variable & Constant & Formula\\
\hline
thermal & energy & temperature & V,N & $\frac{1}{T} = \left(\frac{\partial S}{\partial U}\right)_{V,N}$\\
mechanical & volume & pressure & U,N & $\frac{P}{T} = \left(\frac{\partial S}{\partial V}\right)_{U,N}$\\
diffusive & particles & chemical potential & U,V & $\frac{\mu}{T} = -\left(\frac{\partial S}{\partial N}\right)_{U,V}$
\end{tabular}
\end{table}
\section{Engines and Refrigerators}
\subsection{Heat Engines}
efficiency
\begin{equation}
e \equiv \frac{\text{benefit}}{\text{cost}} = \frac{W}{Q_h}
\end{equation}
$Q_h$ is heat from the hot reservoir with temperature $T_h$, and $Q_c$ from the cold reservoir with temperature $T_c$.
\begin{equation}
Q_h = Q_c + W, \qquad e = 1 - \frac{Q_c}{Q_h}
\end{equation}
From second law
\begin{equation}
S_c \geq S_h \Rightarrow \frac{Q_c}{T_c} \geq \frac{Q_h}{T_h} \Rightarrow \frac{Q_c}{Q_h} \geq \frac{T_c}{T_h}
\label{eq:heatSecond}
\end{equation}
\begin{equation}
 \Rightarrow e \leq 1 - \frac{T_c}{T_h}
\end{equation}
\subsection{Refrigerators}
coefficient of preference:
\begin{equation}
COP \equiv \frac{\text{benefit}}{\text{cost}} = \frac{Q_c}{W}
\end{equation}
From first law $Q_h = Q_c + W$ we get
\begin{equation}
COP = \frac{Q_c}{Q_h - Q_c} = \frac{1}{Q_h/Q_c - 1}
\end{equation}
From second law \eqref{eq:heatSecond} we get
\begin{equation}
COP \leq \frac{1}{T_h/T_c - 1} = \frac{T_c}{T_h - T_c}
\end{equation}

\section{Free Energy and Chemical Thermodynamics}
\subsection{Free Energy as Available Work}
Helmholtz Free Energy: Total energy needed to create the system, minus the heat you can get from the environment for free at temperature $T$. For constant $T$:
\begin{equation}
F = U - TS, \qquad \Delta F = \Delta U - T\Delta S = Q + W - T\Delta S
\end{equation}
For constant $P$ and $T$, the work of a system is Gibbs Free Energy
\begin{equation}
G = H - TS = U - TS + PV, \qquad \Delta G = \Delta U - T\Delta S + P\Delta V = Q+W -T\Delta S + P\Delta V
\end{equation}
\textbf{INSERT PIC ON s151}

\subsection{Thermodynamic Identities}
\begin{equation}
dU = TdS - PdV + \mu dN
\end{equation}
\begin{equation}
dH = dU + PdV + VdP = TdS + VdP + \mu dN
\end{equation}
\begin{equation}
dF = dU - TdS - SdT = - SdT - PdV + \mu dN
\end{equation}
\begin{equation}
S = -\pd{F}{T}{V,N}, \qquad P = -\pd{F}{V}{T,N},\qquad \mu = \pd{F}{N}{T,V}
\end{equation}
\begin{equation}
dG = - SdT + VdP + \mu dN
\end{equation}
\begin{equation}
S = -\pd{G}{T}{P,N}, \qquad V = -\pd{G}{P}{T,N},\qquad \mu = \pd{G}{N}{T,P}
\end{equation}
\subsection{ Free Energy as a Force towards Equilibrium}
\begin{equation}
dS_{total} = dS + dS_R, \qquad dS = \frac{1}{T}dU  + \frac{P}{T}dV - \frac{\mu}{T}dN
\end{equation}
$dS_R = dU_T/T_R$
\begin{equation}
dS_{total} = dS + \frac{1}{T_R}dU_R
\end{equation}
$dU_R = - dU$
\begin{equation}
dS_{total} = dS - \frac{1}{T}dU = -\frac{1}{T}(dU - TdS) =-\frac{1}{T}dF
\end{equation}
This is for constant $T$, $V$ and $N$. For constant $P$
\begin{equation}
dS_{total}=-\frac{1}{T}dG
\end{equation}
\subsection{Extensive and Intensive Quantities}
Double the amount of stuff: Quantities that doubles are extrinsic; those who do not are intensive. \textbf{Extensive}: V, N, S, U, H, F, G, mass. \textbf{Intensive}: T, P, $\mu$, density. Extensive$\times$intensive = extensive. Extensive$\times$extensive = neither. Type$\times$same type = same type. Extensive $+$ intensive is not allowed.
\subsection{Gibbs Free Energy and Chemical Potential}
Given constant $T$ and $P$ we have that
\begin{equation}
\mu = \pd{G}{N}{T,P} \Rightarrow G = N\mu \text{ or } G = \sum_i N_i \mu_i
\end{equation}
\begin{equation}
\Rightarrow\frac{\partial \mu}{\partial P} = \frac{\partial}{\partial P}\frac{G}{N} = \frac{V}{N} = \frac{kT}{P}
\end{equation}
Integrating
\begin{equation}
\mu(T,P) = \mu^\circ(T,P) + kT \ln\frac{P}{P^\circ}
\end{equation}
$P^\circ$ is the atmospheric pressure, and $\mu^\circ$ is $\mu$ at this pressure and can be found in tables for for atmospheric pressures ($\mu = G/N$).  For a mixture $P$ is the partial pressure of that gas.

\subsection{Phase Transformations of Pure Substances}
\textbf{Vapor pressure}: Pressure at which a gas can coexist with its solid or liquid phase. \textbf{Triple point}: Point where all three phases can coexist. \textbf{Critical Point}: No longer a discontinuous change from liquid to gas. \textbf{Curie Temperature}: Temperature where magnetization disappears, so the phase boundary ends at a critical temperature.

\subsection{Diamonds and Graphite}
\textbf{At a given temperature and pressure, the stable phase is always the one with the lower Gibbs free energy}

To find the most stable state, use:
\begin{equation}
\pd{G}{V}{T,N} = V, \qquad \pd{G}{T}{P,N} = -S
\end{equation}
Since graphite has more volume and entropy its Gibbs free energy increases more with pressure and decreases more with temperature. Thus raising the pressure makes diamonds more stable, and with higher temperature higher pressure is needed.
\subsection{Clausius-Clapeyron Relation}
\begin{equation}
G_l = G_g
\end{equation}
at phase boundary. $dG_l = dG_g$ to remain at phase boundary. Thus
\begin{equation}
-S_ldT + V_ldP = -S_gdT + V_gdP
\end{equation}
\begin{equation}
\Rightarrow \frac{dP}{dT} = \frac{S_g - S_l}{V_g - V_l} = \frac{L}{T\Delta V}
\end{equation}
using the (total) latent heat $S_g - S_l = L/T$ and $V_g - V_l = \Delta V$. This is the Clausius-Clapeyron Relation.

\subsection{The van der Waals Model}
\begin{equation}
\left( P + \frac{aN^2}{V^2}\right)(V-Nb) = NkT
\end{equation}
\begin{equation}
\text{total potential pressure} = -\frac{aN^2}{V} \Rightarrow P_{\text{due to p.e.}} = -\frac{d}{dV}\left(-\frac{aN^2}{V}\right) = -\frac{aN^2}{V^2}
\end{equation}
attractive force: $NkT/(V-Nb)$:
\begin{equation}
P = \frac{NkT}{V - Nb} -\frac{aN^2}{V^2}
\end{equation}
\begin{equation}
G = -NkT\ln(V-Nb) + \frac{(NkT)(Nb)}{V-Nb}- \frac{2aN^2}{V} + c(T)
\end{equation}
\begin{equation}
0 = \int_{loop} dG = \int_{loop}\pd{G}{P}{T} dP = \int_{loop} V dP 
\end{equation}
\begin{equation}
V_c = 3Nb,\qquad P_c = \frac{1}{27}\frac{a}{b^2}, \qquad kT_c = \frac{8}{27}\frac{a}{b}
\end{equation}
\subsection{Chemical Equilibrium}
\begin{equation}
0 = dG = \sum_i \mu_i dN_i
\end{equation}
\textbf{Le Chatelier's Principle}: When you disturb a system in equilibrium,it will respond in a way that partially offsets the disturbance.
\subsection{Ionization of Hydrogen}
\begin{equation}
H \leftrightarrow p + e
\end{equation}
Saha equation:
\begin{equation}
\frac{P_p}{P_H} = \frac{kT}{P_e}\left(\frac{2\pi m_e kT}{h^2}\right)^{3/2}e^{-I/kT}
\end{equation}

\section{Boltzmann Statistics}
\begin{equation}
\text{Boltzmann factor} = e^{-E(s)/kT}
\end{equation}
\begin{equation}
\mathcal{P}(s) = \frac{1}{Z}e^{-E(s)/kT},\qquad Z = \sum_s e^{-E(s)/kT}
\end{equation}
\begin{equation}
\bar{E} = \frac{1}{N}\sum_s E(s)N(s) = \sum_s E(s) \mathcal{P}(s) = \frac{1}{Z}\sum_s E(s) e^{-\beta E(s)}
\end{equation}
with $\beta = 1/kT$
\begin{equation}
\bar{E} = -\frac{1}{Z}\frac{\partial Z}{\partial \beta} = -\frac{\partial}{\partial \beta}\ln Z, \qquad U = N\bar{W}
\end{equation}
\subsection{Rotation of Diatomic Molecules}
\begin{equation}
E(j) = j(j+1)\epsilon, \qquad Z_{rot} = \sum_{j=0}^{\infty}(2j+1)e^{-E(j)/kT} = \sum_{j=0}^{\infty}(2j+1)e^{-j(j+1)\epsilon/kT}
\end{equation}
$(2j+1)$ being the degeneration.
\begin{equation}
Z_{rot} \approx \int_{0}^{\infty}(2j+1)e^{-j(j+1)\epsilon/kT} dj = \frac{kT}{\epsilon}
\end{equation}
and $kT/2\epsilon$ for identical atoms ($N_2$, $O_2$ etc).
\subsection{Equipartition Theorem}
Holds for systems where the energy is in the form of quadratic degrees of freedom $E(q) = cq^2$, where c is a constant and $q$ is some variable (coordinate, momentum, etc).
\begin{equation}
Z = \sum_q e^{-\beta E(q)} = \sum_q e^{-\beta cq^2} = \frac{1}{\Delta q}\sum_q e^{-\beta cq^2} \Delta q 
\end{equation}
\begin{equation}
\Rightarrow \frac{1}{\Delta q}\int_{-\infty}^\infty e^{-\beta cq^2} d q =\frac{1}{\Delta q}\frac{1}{\sqrt{\beta c}}\int_{-\infty}^\infty e^{-x^2} d x = C\beta ^{-1/2}
\end{equation}
\begin{equation}
\Rightarrow \bar{E} = -\frac{1}{Z}\frac{\partial Z}{\partial \beta} = \frac{1}{2}kT
\end{equation}
Lennard-Jones:
\begin{equation}
u(x) = u_0\left[\left(\frac{x_0}{x}\right)^{12} - 2\left(\frac{x_0}{x}\right)^{6}\right]
\end{equation}
\subsection{Maxwell Speed Distribution}
\begin{equation}
v_{rms} = \sqrt{\frac{2kT}{m}}
\end{equation}
\begin{equation}
\mathcal{D}(v) = \left(\frac{m}{2\pi kT}\right)^{3/2}4\pi v^2 e^{-mv^2/2kT},\qquad v_{max} = \sqrt{\frac{2kT}{m}}
\end{equation}
\begin{equation}
\bar{v} = \sum_{\text{all v}} v \mathcal{D}(v)dv = \sqrt{\frac{8kT}{\pi m}}
\end{equation}
By turning this to an integral. This is a distribution to to get probability$(v>x)$ just integrate this from $x$ to $\infty$.
\subsection{Partition Functions and Free Energy}
\begin{equation}
F = -kT\ln Z,\qquad Z = e^{-F/kT}
\end{equation}
\begin{equation}
S = -k\sum_s\mathcal{P}(s) \ln \mathcal{P}(s)
\end{equation}
\subsection{Partition Functions for Composite Systems}
For noninteracting, distinguishable particles
\begin{equation}
Z_{total} = Z_1Z_2\ldots Z_N
\end{equation}
For noninteracting, indistinguishable particles
\begin{equation}
Z_{total} =\frac{1}{N!} Z_1^N
\end{equation}

\subsection{Ideal Gas Revisited}
\begin{equation}
Z =\frac{1}{N!} Z_1^N, \qquad Z_1 = Z_{tr}Z_{int}
\end{equation}
Where $E_{tr}$ is the transitional kinetic energy and $E_{int}$ is the internal energy (rotational, vibrational. etc). Particle in box:
\begin{equation}
\lambda_n = \frac{2L}{n}, \qquad p_n = \frac{h}{\lambda_n},\qquad E_n = \frac{p_n^2}{2m} = \frac{h^2n^2}{8mL^2}
\end{equation}
\begin{equation}
Z_{1D} = \sum_ne^{-E_n/kT} = \sum_n e^{-h^2n^2/8mL^2kT}
\end{equation}
Doing this as an integration we get
\begin{equation}
Z_{1D} = \sqrt{\frac{2\pi mk T}{h^2}}L = \frac{L}{\ell_Q} \Rightarrow \ell_Q = \frac{h}{\sqrt{\pi mkT}} 
\end{equation}
Quantum length.
\begin{equation}
Z_{tr} = \frac{L_x}{\ell_Q}\frac{L_y}{\ell_Q}\frac{L_z}{\ell_Q} = \frac{V}{v_Q}
\end{equation}
\begin{equation}
\Rightarrow Z_1 = \frac{V}{v_Q}Z_{int}\Rightarrow Z = \frac{1}{N!}\left(\frac{VZ_{int}}{v_Q}\right)^N,\qquad \ln Z = N( \ln V + \ln Z_{int} - \ln N - \ln v_Q + 1)
\end{equation}
\begin{equation}
U = \frac{\partial}{\partial \beta} Z =  U_{int} + \frac{3}{2}NkT 
\end{equation}
\begin{equation}
C_V = \frac{\partial U}{\partial T} = \frac{\partial U_{int}}{\partial T} + \frac{3}{2}Nk
\end{equation}
\begin{equation}
F = -kT\ln Z = -NkT(\ln V - \ln N - \ln v_Q + 1) + F_{int}
\end{equation}
\begin{equation}
S = Nk\left[\ln \left(\frac{V}{Nv_Q}\right) + \frac{5}{2}\right] - \frac{\partial F_{int}}{\partial T}, \qquad P = \frac{NkT}{V},\qquad \mu = -kT\ln \frac{V Z_{int}}{Nv_Q}
\end{equation}



\end{document}

